\documentclass{article}
\usepackage[utf8]{inputenc}

\title{PS9}
\author{Daniel Carpenter }
\date{March 2020}

\begin{document}

\maketitle

\section{Answers to Questions 6 - 8}
    \subsection{Dimensions of housing.train}
       \begin{center}
        404 x 450
        \end{center}
    
    \subsection{}
            \begin{table}[!htbp] \centering 
              \caption{Comparison of Machine Learning Models} 
              \label{} 
            \begin{tabular}{@{\extracolsep{5pt}} cccc} 
            \\[-1.8ex]\hline 
            \hline \\[-1.8ex] 
             & LASSO & Ridge Reg. & Net Elastic \\ 
            \hline \\[-1.8ex] 
            Lambda & $0.006$ & $0.095$ & $0.014$ \\ 
            In-Sample RMSE & $0.155$ & $0.150$ & $0.165$ \\ 
            Out-of-Sample RMSE & $0.155$ & $0.152$ & $0.171$ \\ 
            \hline \\[-1.8ex] 
            \end{tabular} 
            \end{table} 
    
    \subsection{}
        Since Alpha is around 0.5, it suggests that either model has similar certainty of results. If the model were closer to zero, we would use Ridge Regression.
    
    \subsection{}
        The Net Elastic model seem balanced, so the bias-variance trade off should not be an issue. The two models yield relatively similar RSME's.
        \\ \\
        A linear regression would not be an effective way to estimate this model. Since we have more variables that rows, the model would suffer from multi-colinearity.

\end{document}
