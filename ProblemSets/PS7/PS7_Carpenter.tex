\documentclass{article}
\usepackage[utf8]{inputenc}

\title{PS7_Carpenter}
\author{Daniel Carpenter }
\date{March 2020}

\begin{document}

\maketitle

\section{Project Update}
    Currently, my decision stands to replicate a known stock portfolio optimization \\
    model that is dynamic to any chosen stock. I will likely need to create this \\
    model in julia with the JuMP file. The calculations are extensive in matrix \\
    algebra and linear optimization. Overall, I know that this idea is possible, \\
    but I will likely need to come see you for ideas on how to manipulate some \\
    of the tables to allow for the optimization. I have created this model in \\
    excel, but it is a bear to update and keep up with. If I can create the \\
    framework, then updating this will allow the user to easily update their \\
    portfolio on a day-to-day basis. This project will not only create value \\
    for me, but could also be of extreme use for other users. This model is \\
    one that many individuals on Wall Street use. The end result would be to \\
    create a method that accepts stock inputs, then computing the optimal \\
    allocation of each stock for your portfolio. The final optimization \\
    calculations rely on constrained optimizations that JuMP easily handles. \\
    If their is something similar to the JuMP function in R, I may use that \\
    because I have already set up some of the framework. I will plan to swing \\
    by your office soon.

\newpage
\section{PS7 Answers}
% Table created by stargazer v.5.2.2 by Marek Hlavac, Harvard University. E-mail: hlavac at fas.harvard.edu
% Date and time: Mon, Mar 09, 2020 - 3:58:41 PM
\begin{table}[!htbp] \centering 
  \caption{} 
  \label{} 
\begin{tabular}{@{\extracolsep{5pt}}lccccccc} 
\\[-1.8ex]\hline 
\hline \\[-1.8ex] 
Statistic & \multicolumn{1}{c}{N} & \multicolumn{1}{c}{Mean} & \multicolumn{1}{c}{St. Dev.} & \multicolumn{1}{c}{Min} & \multicolumn{1}{c}{Pctl(25)} & \multicolumn{1}{c}{Pctl(75)} & \multicolumn{1}{c}{Max} \\ 
\hline \\[-1.8ex] 
\hline \\[-1.8ex] 
\end{tabular} 
\end{table}

I am unsure why the table did not include any information



\begin{table}[!htbp] \centering 
  \caption{Regression Results from Finding Missing Values} 
  \label{} 
\begin{tabular}{@{\extracolsep{5pt}}lD{.}{.}{-3} D{.}{.}{-3} } 
\\[-1.8ex]\hline 
\hline \\[-1.8ex] 
 & \multicolumn{2}{c}{\textit{Dependent variable:}} \\ 
\cline{2-3} 
\\[-1.8ex] & \multicolumn{2}{c}{logwage} \\ 
\\[-1.8ex] & \multicolumn{1}{c}{(1)} & \multicolumn{1}{c}{(2)}\\ 
\hline \\[-1.8ex] 
 hgc & 0.062^{***} & 0.050^{***} \\ 
  & (0.005) & (0.004) \\ 
  & & \\ 
 collegenot college grad & 0.145^{***} & 0.168^{***} \\ 
  & (0.034) & (0.026) \\ 
  & & \\ 
 tenure & 0.050^{***} & 0.038^{***} \\ 
  & (0.005) & (0.004) \\ 
  & & \\ 
 I(tenure$\hat{\mkern6mu}$2) & -0.002^{***} & -0.001^{***} \\ 
  & (0.0003) & (0.0002) \\ 
  & & \\ 
 age & 0.0004 & 0.0002 \\ 
  & (0.003) & (0.002) \\ 
  & & \\ 
 marriedsingle & -0.022 & -0.027^{**} \\ 
  & (0.018) & (0.014) \\ 
  & & \\ 
 Constant & 0.534^{***} & 0.708^{***} \\ 
  & (0.146) & (0.116) \\ 
  & & \\ 
\hline \\[-1.8ex] 
Observations & \multicolumn{1}{c}{1,669} & \multicolumn{1}{c}{2,229} \\ 
R$^{2}$ & \multicolumn{1}{c}{0.208} & \multicolumn{1}{c}{0.147} \\ 
Adjusted R$^{2}$ & \multicolumn{1}{c}{0.206} & \multicolumn{1}{c}{0.145} \\ 
Residual Std. Error & \multicolumn{1}{c}{0.344 (df = 1662)} & \multicolumn{1}{c}{0.308 (df = 2222)} \\ 
F Statistic & \multicolumn{1}{c}{72.917$^{***}$ (df = 6; 1662)} & \multicolumn{1}{c}{63.973$^{***}$ (df = 6; 2222)} \\ 
\hline 
\hline \\[-1.8ex] 
\textit{Note:}  & \multicolumn{2}{r}{$^{*}$p$<$0.1; $^{**}$p$<$0.05; $^{***}$p$<$0.01} \\ 
\end{tabular} 
\end{table} 

The intercept seemed to change the most among estimations.

\end{document}
